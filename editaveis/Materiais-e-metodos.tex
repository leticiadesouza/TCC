\chapter{Materiais e Métodos}
\label{ch:metodologia}
\section{Considerações Iniciais}
Neste capítulo é apresentada a classificação da pesquisa e o plano metodológico adotado para o alcance do objetivo desta pesquisa, isto é: \textit{Obter perfis de testadores utilizando características e experiência do domínio para Atribuição automática de casos de teste, utilizando a equipe de transformação de serviços do governo do ITRAC - Information Technology and Application Center}.

Com isso, apresenta-se o planejamento das fases do plano metodológico, seguido do diagnóstico do objeto de estudo.

\section{Plano Metodológico Adotado}
\label{sec:planMetodologico}

O plano metodológico adotado neste trabalho compreende quatro fases básicas: planejamento da pesquisa; coleta de dados; análise dos dados; e relato dos resultados, como apresentado  na Figura \ref{fig:PlanoGeral} abaixo:

        \begin{figure}[H]
          \centering
          \includegraphics[width=6cm]{figuras/planoGeral.png}
          \caption{Plano metodológico adotado (Fonte: Elaborado pela Autora.)}
          \label{fig:PlanoGeral}

        \end{figure}

O trabalho foi desenvolvido no contexto do laboratório ITRAC, junto ao time de testadores envolvidos nos testes dos serviços transformados para o ME. Sendo assim, o objeto de estudo é o time de testadores do ITRAC.

Na fase de coleta de dados os procedimentos de pesquisa empregados foram: pesquisa documental; pesquisa bibliográfica; e a prototipação. Um questionário também foi elaborado para aplicação futura dentre os membros da equipe de teste.

A seguir, as subseções caracterizadas pelas fases do plano metodológico serão apresentadas juntamente com uma descrição de cada um dos procedimentos empregados, como apresentado na Figura \ref{fig:fasesPlano}.


        \begin{figure}[H]
          \centering
          \includegraphics[width=16cm]{figuras/fasesPlanoMetodologico.png}
          \caption{Fases do Plano Metodológico adotadas nesta pesquisa. (Fonte: Elaborado pela Autora.)}
          \label{fig:fasesPlano}

        \end{figure}


\subsection{Fase de Planejamento da Pesquisa}

Na fase de Planejamento da Pesquisa foi configurada a elaboração de todo o Capítulo \ref{ch:introducao} de Introdução. Foi definido o tema de pesquisa, pergunta de pesquisa, objetivos e, também, definição e classificação metodológica.

A Figura \ref{fig:PlanPesquisa} apresenta esta fase importante para as definições iniciais da pesquisa.

        \begin{figure}[H]
          \centering
          \includegraphics[width=6.5cm]{figuras/planejamentoPesquisa.png}
          \caption{Planejamento da pesquisa. (Fonte: Elaborado pela Autora.)}
          \label{fig:PlanPesquisa}

        \end{figure}


        % \begin{figure}[H]
        %   \centering
        %   \includegraphics[width=14cm]{figuras/planoMetodologico.png}
        %   \caption{Plano metodológico adotado (Fonte: Elaborado pela Autora.)}
        %   \label{fig:PlanoMetodologico}

        % \end{figure}

\subsection{Fase de Coleta de Dados}

A fase de coleta de dados configurou o recolhimento de boa parte do conteúdo necessário para a consolidação deste trabalho, foram adotados os procedimentos de \textit{pesquisa bibliográfica} e adotados os procedimentos da \textit{pesquisa - ação}.

A Figura \ref{fig:coletaDados} apresenta os procedimentos de coleta de dados e as técnicas adotadas para o presente trabalho.

        \begin{figure}[H]
          \centering
          \includegraphics[width=10cm]{figuras/coletadeDados.png}
          \caption{Coleta de Dados. (Fonte: Elaborado pela Autora.)}
          \label{fig:coletaDados}

        \end{figure}

A análise bibliográfica permitiu com que fosse recolhido embasamentos relevantes sobre os temas relacionados a esta pesquisa. A pesquisa-ação também promoveu uma interação maior entre pesquisados e equipe do objeto de estudo do laboratório.

\section{Pesquisa Bibliográfica}

A partir da pesquisa bibliográfica foi possível recolher embasamentos relevantes sobre Governo Digital, Transformação Digital, Testes Exploratórios, Perfil de Testadores e Sistemas de Recomendação. Cada um dos embasamentos sobre os temas coletados foram apresentados no Capítulo \ref{ch:referencial} de Referencial Teórico.

\subsection{Fase de Pesquisa-Ação}

A estratégia de Pesquisa-Ação neste trabalho foi adotada a partir de uma adaptação da pesquisa-ação proposta por \cite{petersen2008systematic}. Na estratégia são propostas 4 etapas distintas, conforme apresentado na Figura \ref{fig:etapasPesquisaAcao}.

        \begin{figure}[H]
          \centering
          \includegraphics[width=12cm]{figuras/etapasPesquisaAcao.png}
          \caption{Etapas da estratégia de Pesquisa-Ação adotada. (Fonte: Elaborado pela Autora.)}
          \label{fig:etapasPesquisaAcao}

        \end{figure}

É importante ressaltar que a estratégia será empregada em ciclos para a fase de implementação do sistema de recomendação, ou seja, a adaptação da ideia e melhoria da proposta deste trabalho irá permitir uma maior interação com a equipe de testadores de modo que, de maneira colaborativa, se extraia mais informações relevantes para a otimização da recomendação realizada pelo sistema.

\subsection{Diagnóstico}

Para a construção do sistema de recomendação um \textit{diagnóstico} foi realizado para a compreensão do objeto de estudo. Para a caracterização do objeto de estudo, apresentado no Capítulo \ref{ch:referencial}, foram realizadas pesquisas quanto a sistemas de recomendação e perfil de testadores, a fim de entender afundo o \textit{background} teórico que sustenta este trabalho.

\subsection{Planejamento, Ação e Validação}

As estapas de pesquisa-ação, planejamento, ação e avaliação serão levadas em consideração para a construção de instrumentos em ciclos de interação com a equipe, a fim de facilitar o entendimento sobre

A Figura \ref{fig:etapasPesquisaAcaoAdotas} apresenta a estratégia de Coleta de Dados, ressaltando a Pesquisa-ação e respectivas etapas empregadas.

        \begin{figure}[H]
          \centering
          \includegraphics[width=12cm]{figuras/etapasPesquisaAcaoAdotadas.png}
          \caption{Etapas da Pesquisa-Ação adotadas. (Fonte: Elaborado pela Autora.)}
          \label{fig:etapasPesquisaAcaoAdotas}

        \end{figure}

\subsection{Técnicas de Coleta de Dados}

Durante a fase de Coleta de dados foram usadas as técnicas de observação, além de elaboração e futura aplicação de questionário.

\section{Análise de Dados}

Após a fase Coleta de Dados, com a aplicação dos procedimentos de coleta de dados definidos, a fase seguinte será a de Análise dos Dados e Resultados, conforme o planejamento da pesquisa e apresentado na Figura \ref{fig:analiseDados}.

        \begin{figure}[H]
          \centering
          \includegraphics[width=8cm]{figuras/analiseDados.png}
          \caption{Análise de Dados. (Fonte: Elaborado pela Autora.)}
          \label{fig:analiseDados}

        \end{figure}

\section{Resultados}

A fase de resultados corresponde à fase final deste trabalho, onde os resultados obtidos com a execução do TCC2 serão apresentados. A Figura X apresenta a última fase do Plano Metodológico.

        \begin{figure}[H]
          \centering
          \includegraphics[width=8cm]{figuras/resultados.png}
          \caption{Resultados. (Fonte: Elaborado pela Autora.)}
          \label{fig:resultados}

        \end{figure}


\section{Cronograma}

A Figura \ref{fig:cronograma} apresenta o cronograma desde os trabalhos realizados durante o TCC1 até a concretização do sistema de recomendação, no TCC2.

        \begin{figure}[H]
          \centering
          \includegraphics[width=15cm]{figuras/cronograma.png}
          \caption{Cronograma. (Fonte: Elaborado pela Autora.)}
          \label{fig:cronograma}

        \end{figure}

\section{Considerações Finais do Capítulo}

Considerando o foco deste trabalho em definir processos, principalmente, relacionados à atribuição de testes, a metodologia definida surgiu da necessidade de se explicitar uma ampla visão sobre o desenvolver deste trabalho. Desse modo, o capítulo apresentou, detalhadamente, como os passos para se atingir os objetivos da pesquisa foram consolidados.

Além disso, foi apresentado o cronograma deste trabalho, a fim de estabelecer e especificar as atividades que foram realizadas ao longo do desenvolvimento deste TCC. Sendo assim, os passos para os trabalhos a serem realizados também serão atualizados na tabela ao longo do desenvolvimento.
