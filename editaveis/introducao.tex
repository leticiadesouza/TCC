\chapter[Introdução]{Introdução}
%\addcontentsline{toc}{chapter}{Introdução}

A atividade de teste é essencial dentro do contexto da Engenharia de Software objetivando a validação do comportamento
do software e identificação de possíveis problemas de funcionamento. Nas últimas décadas, tem sido estabelecidas
técnicas, critérios, métodos e ferramentas para a produção de software a fim de acompanhar a crescente utilização de
sistemas por parte da maioria das práticas da atividade humana~\cite{maldonado2004introduccao}.

Através da examinação diretamente do software em execução, o teste fornece um \textit{feedback} realista de seu comportamento,
o que o torna o complemento inevitável de outras técnicas de análise e garantia de qualidade na indústria~\cite{bertolino2007software}.

\section{Contextualização}

Através do advento da evolução da tecnologia e expansão da internet surge a expressão \textit{Governo Digital},
que se refere ao uso de tecnologias digitais como uma estratégia de modernização de um Governo, que adota a tecnologia
da informação e comunicação (TIC) para prover serviços~\cite{fang2002government}. O termo \textit{e-Gov} surgiu no final da década de 1990 e cresceu a um tamanho considerável desde então. Nos últimos
anos, deu origem a várias conferências de cunho científico, aumentando seu conteúdo e posição no que se refere a outros
campos de pesquisa e disciplinas~\cite{gronlund2005introducing}.

No Brasil, embora as medidas de modernização do setor público tenham começado a ser adotadas na década de 70, as ênfases se deram apenas com a crise fiscal
dos anos 80, onde a intervenção estatal ficou conhecida como reforma da gestão pública que, aliada as TICs, proporcionou com que os governos no Brasil oferecessem
serviços públicos eletrônicos à população no início da década de 2000~\cite{przeybilovicz2015desenvolvimento}.

Atualmente, o Governo Federal tem oferecido suporte às agências brasileiras para que busquem a transferências de seus serviços
para \textit{serviços digitalizados}, promovendo uma aproximação entre os governos e a sociedade, diminuindo o esforço do governo em
ter que buscar o entendimento sobre as necessidades do cidadão \textit{(citizen-centric)} e aumentando a capacidade de reação
sobre as necessidades determinadas por eles \textit{(citizen-driven)}.

Em vista deste cenário de digitalização do serviço público, é de suma importância garantir a qualidade dos serviços digitalizados
devido aos efeitos nos domínios nacionais e internacionais. De acordo com~\cite{myers2004art}, é preciso que as características
dos projetos sejam levadas em consideração no momento de definição da estratégia de teste para assegurar que a verificação e
a validação sejam economicamente viáveis, analisando a complexidade e quantidade de caminhos envolvidos para que se atinja a
qualidade desejada.

A garantia da qualidade das etapas intermediárias entre requisitos e o produto real pode ser assegurada pela atividade de
verificação através de uma perspectiva específica, avaliando se o software desenvolvido está de acordo com os requisitos
estabelecidos e acordados~\cite{wallace1989software}.

Em cenários nos quais os requisitos não foram inicialmente bem definidos, são instáveis, voláteis ou, até mesmo, inexistentes,
os testadores podem encontrar o problema oráculo~\cite{barr2015oracle}, tornando difícil identificar se o comportamento atual do
serviço digitalizado corresponde ao comportamento esperado ou se esse comportamento está errado.

O problema de testes sem requisitos está relacionado ao contexto deste trabalho. Os serviços utilizados no contexto deste trabalho
não possuem uma documentação formal.

\section{Problema de Pesquisa}

\section{Justificativa}

\section{Objetivos}
\subsection{Objetivo Geral}

Obter perfis de testadores utilizando características e experiência do domínio para alocação automática de casos de teste,
utilizando a equipe de transformação de serviços do governo do ITRAC - Information Technology and Application Center.

\subsection{Objetivos Específicos}

\begin{itemize}
		\item Identificar soluções para o problema de SLAM em um contexto simplificado;
		\item Propor adaptações para o contexto de robôs simples, na Robótica Educacional;
		\item Implementar adaptação;
		\item Analisar viabilidade da resolução do problema de SLAM no contexto limitado da Robótica Educacional.
	\end{itemize}

\section{Organização do Trabalho}
