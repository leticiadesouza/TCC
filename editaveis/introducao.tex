\chapter[Introdução]{Introdução}
%\addcontentsline{toc}{chapter}{Introdução}
\label{ch:introducao}

A atividade de teste é essencial dentro do contexto da Engenharia de Software objetivando a validação do comportamento
do software e identificação de possíveis problemas de funcionamento. Nas últimas décadas, tem sido estabelecidas
técnicas, critérios, métodos e ferramentas para a produção de software a fim de acompanhar a crescente utilização de
sistemas por parte da maioria das práticas da atividade humana~\cite{maldonado2004introduccao}.

Através da examinação diretamente do software em execução, o teste fornece um \textit{feedback} realista de seu comportamento,
o que o torna o complemento inevitável de outras técnicas de análise e garantia de qualidade na indústria~\cite{bertolino2007software}.

\section{Contextualização}

Através do advento da evolução da tecnologia e expansão da internet surge a expressão \textit{Governo Digital},
que se refere ao uso de tecnologias digitais como uma estratégia de modernização de um Governo, que adota a tecnologia
da informação e comunicação (TIC) para prover serviços~\cite{fang2002government}. O termo \textit{e-Gov} surgiu no final da década de 1990 e cresceu a um tamanho considerável desde então. Nos últimos
anos, deu origem a várias conferências de cunho científico, aumentando seu conteúdo e posição no que se refere a outros
campos de pesquisa e disciplinas~\cite{gronlund2005introducing}.

No Brasil, embora as medidas de modernização do setor público tenham começado a ser adotadas na década de 70, as ênfases se deram
apenas com a crise fiscal dos anos 80, onde a intervenção estatal ficou conhecida como reforma da gestão pública que, aliada as
TICs, proporcionou com que os governos no Brasil oferecessem serviços públicos eletrônicos à população no início da década de
2000~\cite{przeybilovicz2015desenvolvimento}.

Atualmente, o Governo Federal tem oferecido suporte às agências brasileiras para que busquem a transferências de seus serviços
para \textit{serviços digitalizados}, promovendo uma aproximação entre os governos e a sociedade, diminuindo o esforço do governo
em ter que buscar o entendimento sobre as necessidades do cidadão \textit{(citizen-centric)} e aumentando a capacidade de reação
sobre as necessidades determinadas por eles \textit{(citizen-driven)}.

Diante deste cenário, foram publicados decretos que definiram a \textit{Política de Governança Digital}, que possui como objetivo
ampliar e simplificar o acesso dos cidadãos brasileiros aos serviços públicos digitais, inclusive por meio
de dispositivos móveis, e a \textit{Plataforma de Cidadania Digital}, ambos relacionados ao âmbito da Administração Pública Federal
(APF) e de responsabilidade do Ministério do Planejamento, Desenvolvimento e Gestão (MPDG) \cite{BRASIL} que, atualmente,
correspondem ao Ministério da Economia(ME).

Uma das iniciativas do MP envolveu a criação de um programa de \textit{Trans de Serviços Públicos} denominado
\textit{Kit de Transformação},composto de seis fases independentes entre si: \textit{Questione}, \textit{Personalize},
\textit{Reinvente}, \textit{Facilite},\textit{Integre} e \textit{Comunique} \cite{BRASIL2017}.

O presente trabalho está relacionado à fase \textit{Facilite}, esta fase provê recursos e ferramentas na tentativa de simplificar
e digitizar serviços a partir de uma solução tecnológica. A conclusão dessa fase é atingida quando as simplificações no serviço
estiverem implantadas.

Em vista deste cenário de digitalização do serviço público, é de suma importância garantir a qualidade dos serviços digitalizados
devido aos efeitos nos domínios nacionais e internacionais. De acordo com~\cite{myers2004art}, é preciso que as características
dos projetos sejam levadas em consideração no momento de definição da estratégia de teste para assegurar que a verificação e
a validação sejam economicamente viáveis, analisando a complexidade e quantidade de caminhos envolvidos para que se atinja a
qualidade desejada.

A garantia da qualidade das etapas intermediárias entre requisitos e o produto real pode ser assegurada pela atividade de
verificação através de uma perspectiva específica, avaliando se o software desenvolvido está de acordo com os requisitos
estabelecidos e acordados~\cite{wallace1989software}.

Nos cenários nos quais os requisitos não foram inicialmente bem definidos, são instáveis, voláteis ou, até mesmo, inexistentes,
os testadores podem encontrar o problema do oráculo~\cite{barr2015oracle}, tornando difícil a identificação do comportamento atual
do serviço digitalizado corresponde ao comportamento esperado, ou se esse comportamento está errado. A problemática em se testar sem
requisitos está relacionada ao contexto deste trabalho.

O processo de validação dos serviços digitalizados do governo brasileiro possuem uma natureza crítica. Os serviços utilizados no
contexto deste trabalho não possuem uma documentação formal de requisitos e o código-fonte não é acessível à equipe de teste, o que
impossibilita a extração de informações. Dado este contexto, foi desenvolvido um processo de validação sistematizado que garante
a qualidade da digitalização dos serviços, este processo foi definido baseado na estratégia proposta por \cite{elcock2006testing} e
no conceito de Teste Exploratório, sugerido por \cite{whittaker2009exploratory}, usando a metáfora do turista como uma estratégia
para guiar a descoberta de requisitos e regras de negócios, além de sistematizar a criação e execução dos casos de teste.

\section{Problema de Pesquisa}

A forma como os casos de teste são criados dentro de uma equipe pode implicar na produtividade a depender do nível de expertise dos
testadores. Dentro do presente contexto, sabe-se que a experiência dos testadores relacionados aos serviços disponibilizados
pode variar bastante, devido às variações de nível de escolaridade, experiência com testes, tipos de testes, técnicas já utilizadas
e outras variáveis que serão tratadas mais afundo ao longo deste trabalho.

Por este motivo, a problemática relacionada a estes pontos está relacionada com a produtividade das equipes de teste manual quando
voltadas à digitalização de serviços públicos. Há algumas questões de pesquisas relacionadas a este trabalho, que serão abordadas a partir do modelo
\textit{Goal Question Metric} (GQM).

Sendo assim, a pergunta geral de pesquisa definida neste trabalho é:

\textit{"Como maximizar a produtividade de uma equipe de teste manual no contexto de digitalização de serviços públicos?"}

\section{Justificativa}

Há indícios de que o perfil do testador tenha algum impacto na aplicação das tours utilizadas na metáfora do turista proposta por
\cite{whittaker2009exploratory}, visto que os casos de teste gerados e suas sequências variam muito a depender do testador.
O entendimento desse perfil e suas principais características pode trazer alguns benefícios para uma equipe de testes.

Normalmente, casos de teste são alocados por gerentes para que sejam executados pelos testadores que estiverem disponíveis em 
determinado momento. Em vista disso, uma das formas de se maximizar a produtividade em uma equipe de testes é alocar casos de teste de acordo com o 
perfil de testadores, embora a alocação de teste manual não seja uma tarefa trivial \cite{miranda2012recommender}.

Sistemas de recomendação têm sido propostos para alocar tarefas a perfis específicos baseado em análises de alocações anteriores, como 
comprovam os trabalhos de \cite{anvik2006should} e \cite{miranda2012recommender}.


\section{Objetivos}
\subsection{Objetivo Geral}

Obter perfis de testadores utilizando características e experiência do domínio para alocação automática de casos de teste,
utilizando a equipe de transformação de serviços do governo do ITRAC - Information Technology and Application Center.

\subsection{Objetivos Específicos}

\begin{itemize}
		\item Identificar a interferência que o perfil do testador tem na eficiência dos testes;
		\item Propor uma atribuição automática de casos de teste adequada ao perfil do testador; 
		\item Analisar dados registrados nos ciclos de teste;
	\end{itemize}

\section{Organização do Trabalho}

Este trabalho de conclusão de curso está organizado nos seguintes capítulos:

\begin{itemize}
    \item \textbf{Capítulo \ref{ch:introducao} - Introdução:} apresenta Contextualização, justificativa, problema de pesquisa, justificativa e objetivos.
    \item \textbf{Capítulo \ref{ch:referencial} - Referencial teórico:} descreve os conceitos que fundamentam o trabalho reunindo conhecimento necessário para que se compreenda a pesquisa realizada. O capítulo é subdividido nas seções \textit{Governo Digital}, \textit{Testes Exploratórios} e \textit{Sistema de Recomendação};
    \item \textbf{Capítulo \ref{ch:suporte} - Suporte tecnológico:} apresenta as ferramentas que suportarão as atividades de desenvolvimento de software, gerenciamento, documentação, dentre outras.
    \item \textbf{Capítulo \ref{ch:proposta} - Proposta:} detalha os procedimentos de seleção de padrões de projeto e classificação dos mesmos.
    \item \textbf{Capítulo \ref{ch:metodologia} - Metodologia:} estabelece os procedimentos a serem seguidos do início até a conclusão do trabalho;
    \item \textbf{Capítulo \ref{ch:resultados} - Resultados parciais:} apresenta os resultados alcançados durante o TCC1.
    \item \textbf{Capítulo \ref{ch:consideracoes} - Considerações finais:} relata o status do trabalho alcançado até a execução do TCC1 e os resultados esperados para o TCC2.
\end{itemize}