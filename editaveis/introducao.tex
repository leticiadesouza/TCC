\chapter[Introdução]{Introdução}
%\addcontentsline{toc}{chapter}{Introdução}

A atividade de teste é essencial dentro do contexto da Engenharia de Software objetivando a validação do comportamento
do software e identificação de possíveis problemas de funcionamento. Nas últimas décadas, tem sido estabelecidas
técnicas, critérios, métodos e ferramentas para a produção de software a fim de acompanhar a crescente utilização de
sistemas por parte da maioria das práticas da atividade humana.\cite{maldonado2004introduccao}.

	

\section{Contextualização}

\section{Problema de Pesquisa}

\section{Justificativa}

\section{Objetivos}
\subsection{Objetivo Geral}

Obter perfis de testadores utilizando características e experiência do domínio para alocação automática de casos de teste,
utilizando a equipe de transformação de serviços do governo do ITRAC - Information Technology and Application Center.

\subsection{Objetivos Específicos}

\begin{itemize}
		\item Identificar soluções para o problema de SLAM em um contexto simplificado;
		\item Propor adaptações para o contexto de robôs simples, na Robótica Educacional;
		\item Implementar adaptação;
		\item Analisar viabilidade da resolução do problema de SLAM no contexto limitado da Robótica Educacional.
	\end{itemize}

\section{Organização do Trabalho}
