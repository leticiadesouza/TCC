\section{Problema}

A forma como os casos de teste são criados dentro de uma equipe pode implicar na produtividade a depender do nível de expertise dos
testadores. Dentro do presente contexto, sabe-se que a experiência dos testadores relacionados aos serviços disponibilizados
pode variar bastante, devido às variações de nível de escolaridade, experiência com testes, tipos de testes, técnicas já utilizadas
e outras variáveis que serão tratadas mais afundo ao longo deste trabalho.

Por este motivo, a problemática relacionada a estes pontos está relacionada com a produtividade das equipes de teste manual quando voltadas à digitalização de serviços públicos. 

%Há algumas questões de pesquisas relacionadas a este trabalho, que serão abordadas a partir do modelo
%\textit{Goal Question Metric} (GQM), na seção X desse trabalho, especificamente.

Sendo assim, a pergunta de pesquisa definida neste trabalho é:

\textit{``Como maximizar a produtividade de uma equipe de teste manual no contexto de digitalização de serviços públicos?''}