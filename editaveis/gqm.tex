\chapter[gqm]{Goal Question Metric}

Neste capítulo é descrito o Goal Question Metric (GQM) e, com base na análise do GQM, a intenção é verificar a 
existência de correlação entre a eficiência e as diferentes variáveis que compóem o perfil dos testadores. 
O perfil será identificado com base em diferentes questões respondidas pelos testadores realizadas através de um
questionário digital. Respondidas as questões e coletados os dados sobre a eficiência, é possível a realização 
de testes de correlação para identificar se há alguma relação entre as variáveis do perfil e a eficiência nos 
testes de determinado testador ou da equipe de testes. 

\section{Goals}

O objetivo desta pesquisa é analisar como maximizar a produtividade de uma equipe de teste manual no contexto de
digitalização de serviços públicos. 

\textbf{G1}: Aaprimorar a atribuição de tarefas de teste com base no perfil do testador.

\textbf{Question 1}: Qual a eficiência da alocação de casos de teste por parte da equipe?
M1: Número de casos de teste gerados pela equipe
M2: Número de falhas identificadas pela equipe
M3: M2/M1

\textbf{Question 2}: Qual a eficiência da alocação de casos de teste por parte do testador?

\textbf{M1}: Número de casos de teste gerados pelo testador

\textbf{M4}: Número de falhas identificadas por testador

\textbf{Question 3}: Qual o impacto do perfil do testador na eficiência dos testes? No caso do estudo em questão, 

o perfil será avaliado frente as seguintes variáveis: 

\begin{itemize}
    \item tempo de experiência em teste (TET); 
    \item familiaridade com teste exploratório (FTE);
\end{itemize}

\textbf{M5}: Variação entre a eficiência de cada testador, onde, eficiência = Número de defeitos/número de casos de teste 

\textbf{Question 4}: Qual o grau de satisfação do testador durante o ciclo de teste? 

\textbf{M6}: Grau de satisfação (0 a 5) 
\textbf{Question 5} Qual a eficiência dos testes em relação a cada tour?
\textbf{M6 (tour 1)} = M1(tour 1)/M4(tour1) (análise realizada por tour, tanto p/equipe quanto p/ testador individual) 

