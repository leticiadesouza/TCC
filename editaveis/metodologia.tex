\chapter[Metodologia]{Metodologia}

Este capítulo tem por objetivo descrever a metodologia utilizada além dos fluxos seguidos para atingir o obejtivo da pesquisa. 


\section{Metodologias de Pesquisa}

\section{Escolhas Metodológicas}

% lembrar de descrever sobre a acaracterística de ser uma pesquisa de desenvolvimento e revisão bibliográfica.

\subsection{Processo de revisão sistemática}

Revisões sistemáticas de literatura (RSL) focam na problemática de agregar o máximo de evidências empíricas que se pode ter utilizando uma variedade de técnicas em contextos diferentes, através do problema de agregar evidências empíricas.

Evidências impíricas podem ser obtidas através do uso de uma variedade de técnicas, e em contextos, de acordo com \textit{kitchenham2009systematic} que, em seu processo de revisão sistemática, propõe três etapas: planejamento da revisão, condução da revisão e documentação da revisão, as quais são utilizadas nessa RSL. 

\par Além disso, \textit{khan2001undertaking} recomenda para relatos de RSL o uso de uma estrutura abstrata utilizando contexto, objetivos, metodologia, resultados e conclusões, o presente artigo se organiza seguindo essa mesma linha de estrutura abstrata.

\subsection{Planejamento}
\par A revisão sistemática utilizou as bases \textit{Scopos}, \textit{ACM} e \textit{IEEE}, utilizando critérios de seleção para filtrar a inclusão e exclusão dos artigos, tais como:

\begin{itemize}
    \item Os artigos devem estar escritos em inglês ou português
    \item Artigos referentes a testes exploratórios e validação de software
    \item Artigos de acesso gratuito, ou de acesso concedido para alunos da Universidade
    \item Artigos que envolvem o teste exploratório como uma solução para o desenvolvimento
\end{itemize}

\subsubsection{Objetivos e Questões de Pesquisa}

\par Através da busca pelo que tem sido publicado sobre sistemas de recomendação no contexto de testes manuais de software, enxerga-se a possibilidade de entender o que se pode obter de mais relevante sobre tal domínio para embasar o trabalho proposto.

\par Dessa forma, objetivando analisar o atual quadro de pesquisa em sistemas de recomendação dentro do contexto de testes  manuais de software, foram levantadas as seguintes questões de pesquisa, derivadas do GQM \textit{Goal Question Metric}: 

\begin{itemize}
    \item Q1. 
    \item Q2. 
    \item Q3. 
\end{itemize}

\par Estudar as problemáticas levantadas pelas questões definidas permitiu com que se realizasse uma pesquisa bibliográfica com o objetivo de identificar diferentes linhas de pesquisa na área. 

\subsubsection{Estratégia de pesquisa}

Assim como proposto por \textit{kitchenham2009systematic}, para a realização dessa revisão sistemática foi utilizado o processo que, embora tenha tido os títulos das seções adaptados, segue três etapas: planejamento da revisão, condução da revisão e a documentação da revisão. 

Durante as pesquisas relacionadas ao processo de revisão sistemática, observou-se um conceito interessante de \textit{gold standard} definido por \textit{biolchini2005systematic} como uma espécie de teste controlado aleatório realizado às cegas. Nessa forma de experimentar, é feita uma analogia na qual tanto o doutor quanto o paciente não sabem sobre o tratamento pelo qual o paciente precisa ser submetido e, a razão disso é que as expectativas do doutor e do paciente não podem influenciar os resultados, justificando o fato de ser algo "feito às cegas". 

Assim como a analogia referenciada no parágrafo anterior, funciona o protocolo experimental que, segundo \textit{biolchini2005systematic}, são impossíveis de serem realizados em experimentos de engenharia de software que dependem de um sujeito que executa uma tarefa intensiva. 

Embora na Engenharia de Software não seja possível obter esse \textit{gold standard} em grande parte das RSLs, o conceito \textit{quasi-gold standard}, criado por \textit{zhang2011identifying}, surge para representar um subconjunto do \textit{gold standard}, que é evoluído à medida que os ciclos de busca vão acontecendo, procurando se aproximar do \textit{gold standard}.

Através do \textit{quasi-gold standard} é possível definir valores de precisão e sensitividade da busca realizada, de modo que a avaliação da busca, após analisada, pode ser verificada e evidenciar a necessidade de um refinamento da \textit{string}. \textit{zhang2011identifying} define a precisão e a sensitividade assim:

\begin{itemize}

    \item $precisao = \frac{ERO}{EO}$ 
	\item $sensitividade = \frac{ERO}{TER}$

\end{itemize}

onde \textit{ERO} = número de estudos relevantes obtidos,
		
	  \textit{EO} = número de estudos obtidos e

	   \textit{TER} = número total de estudos relevantes.

Através desse conceito é possível criar critérios de avaliação para determinar a relevância e qualidade das buscas realizadas, como foi relizado na seção 2.2.

Responder às questões levantadas exigiu as buscas utilizando as strings apresentadas na próxima seção (3) e, afim de executar o planejamento, a \textit{string} escolhida foi utilizada nas três bases escolhidas, sendo que, todos os artigos resultantes da busca foram coletados para análise. 

\section{Execução}
A primeira \textit{string} utilizada foi "\textit{"recommender system" AND "exploratory test"}", que originou o seguinte resultado:

\begin{table}[H]
\begin{tabular}{c|c|c|c}
\textit{String}                                                                   & \textit{Scopus} & \textit{ACM} & \textit{IEEE} \\ \hline
\begin{tabular}[c]{@{}c@{}}"recommender" AND\\ "exploratory test"\end{tabular} & 82               & 8           & 357           
\end{tabular}
\end{table}

A \textit{string} de busca inicial resultou um vasto número de artigos que envolviam os termos escolhidos, de modo que as questões de pesquisa não eram abordadas em muitos dos resultados. A análise de relevância de cada artigo obtido com essa \textit{string} foi realizada superficialmente e, percebendo a divergência dos assuntos com os objetivos de pesquisa, foi necessário refinar a \textit{string}, utilizando de termos relacionados ao contexto dos testes exploratórios. 

No próximo ciclo de busca foi incluída a palavra-chave citada, que desencadeou a seguinte \textit{string}: \textit{"exploratory test" AND "allocation" AND "recommender system"}. Entretanto, os resultados obtidos nas bases foram muito irrelevantes, resultando, inclusive, artigos que o primeiro ciclo de busca já havia retornado. 

Sendo assim, a \textit{string} sofreu modificações novamente para se adaptar às questões levantadas. A partir desse resultado, optou-se por retirar o conceito de \textit{feature testing}, substituindo pelo conceito de \textit{usability}, formando a seguinte \textit{string}: \textit{"exploratory test" AND "usability" AND development}. 

O objetivo da inclusão desse termo, foi obter as informações relacionadas à validação do \textit{front-end} do software, visto que, nos ciclos de busca anteriores, o conceito de testes exploratórios está bastante relacionado à validação do \textit{front-end}. Entretanto, a inclusão do termo \textit{usability} não gerou resultados muito significativos, visto que, os artigos mais relevantes à pesquisa, já haviam sido retornados à partir de outras \textit{strings} utilizadas. 

Os ciclos de busca anteriores mostraram que o conceito de testes exploratórios está bastante relacionado ao processo de validação de software, visto que é uma técnica de teste caixa preta. Desse modo, em um novo ciclo de busca, foi incluso o termo \textit{validation}. 

Além disso, optou-se pela generalização de alguns termos na string a partir do uso de algumas funcionalidades proporcionadas pelas bases como a utilização do ’*’, que significa que quaisquer caracteres precedidos dos caracteres anteriores ao ’*’ serão
considerados. 

\begin{table}[H]
\begin{tabular}{c|c|cc}
\multirow{2}{*}{\textbf{String}} & \multirow{2}{*}{\textit{\textbf{\begin{tabular}[c]{@{}c@{}}Fonte de \\ Busca\end{tabular}}}} & \multicolumn{2}{c}{\textit{\textbf{Resultados}}}          \\ \cline{3-4} 
                                 &                                                                                              & \multicolumn{1}{c|}{\textbf{Total}} & \textbf{Relevantes} \\ \hline
\multirow{3}{*}{Inicial}         & \textit{Scopus}                                                                              & \multicolumn{1}{c|}{82}             & 14                  \\ \cline{2-4} 
                                 & \textit{ACM}                                                                                 & \multicolumn{1}{c|}{8}              & 5                   \\ \cline{2-4} 
                                 & \textit{IEEE}                                                                                & \multicolumn{1}{c|}{357}            & x                   \\ \hline
\multirow{3}{*}{Refinada}        & \textit{Scopus}                                                                              & \multicolumn{1}{c|}{3}              & 0                   \\ \cline{2-4} 
                                 & \textit{ACM}                                                                                 & \multicolumn{1}{c|}{3}              & 3                   \\ \cline{2-4} 
                                 & \textit{IEEE}                                                                                & \multicolumn{1}{c|}{19}             & 6                  
\end{tabular}
\end{table}

Realizando todas estas buscas, pôde-se aplicar o conceito de \textit{sensitividade} e \textit{precisão}, utilizando, além da soma dos valores da tabela, o número total de artigos considerados relevantes  demonstrando o seguinte resultado:

\begin{itemize}

    \item $sensitividade = \frac{12}{25}$ 
	\item $precisao = \frac{12}{28}$

\end{itemize}

À partir desse resultado, percebe-se uma \textit{sensitividade} de 48\% e uma precisão de 42\%, finalizando os ciclos de busca com 28 artigos relevantes selecionados.  

