%\part{Aspectos Gerais}
\chapter[Referencial Teórico]{Referencial Teórico}
Visando facilitar o entendimento acerca dos principais temas desta pesquisa, este capítulo define os aspectos e elementos relacionados a sistema de
recomendação, perfil de testadores, testes exploratórios e governo digital.
%---------------- Governo Digital ---------------------%
\section{Governo Digital}

Através do advento da evolução da tecnologia e expansão da internet surge a expressão \textit{Governo Digital},
que se refere ao uso de tecnologias digitais como uma estratégia de modernização de um Governo, que adota a tecnologia
da informação e comunicação (TIC) para prover serviços~\cite{fang2002government}. O termo \textit{e-Gov} surgiu no final da década de 1990 e cresceu a um tamanho considerável desde então. Nos últimos
anos, deu origem a várias conferências de cunho científico, aumentando seu conteúdo e posição no que se refere a outros
campos de pesquisa e disciplinas~\cite{gronlund2005introducing}.

A OECD (Organisation for Economic Co-operation and Development) usa a expressão \textit {Governo Eletrônico} como o uso de Tecnologias de Informação e
Comunicação pelo governo como uma ferramenta para melhorar as atividades do governo, especialmente o uso da Internet. A expressão
\textit{Digital Government}, por outro lado, refere-se ao uso de tecnologias digitais como parte integrante das estratégias de modernização de um governo,
a fim de adicionar valor público. A integração de novas tecnologias no cotidiano faz com que as expectativas dos cidadãos sobre suas relações com os governos
mudem \cite{oecd}. Em consequência disso, os governos estão introduzindo novas abordagens de governança pública que atendam às atuais necessidades de
cidadãos e empresas, trazendo a digitização de serviços, que tem por objetivo a otimização de serviços e processos de forma que os mesmos estejam de fato
integrados ao meio digital.

No Brasil, embora as medidas de modernização do setor público tenham começado a ser adotadas na década de 70, as ênfases se deram apenas com a crise fiscal
dos anos 80, onde a intervenção estatal ficou conhecida como reforma da gestão pública que, aliada as TICs, proporcionou com que os governos no Brasil oferecessem
serviços públicos eletrônicos à população no início da década de 2000~\cite{przeybilovicz2015desenvolvimento}.

Em 2016 o lançamento da Estratégia de Governança Digital da Administração Pública Federal \cite{BRASIL}, iniciou um novo paradigma de gestão pública. A EGD, finalizada em 2019,
“define os objetivos estratégicos, metas e indicadores da Política de Governança Digital” e afirma que seu principal desafio é cultural.


%---------------- Testes Exploratórios-----------------%

\section{Testes exploratórios}

%---------------- Sistema de Recomendação--------------%

\section{Sistema de Recomendação}
