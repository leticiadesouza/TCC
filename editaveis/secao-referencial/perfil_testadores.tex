\section{Perfil de Testadores}
\label{sec:perfil_testadores}


A atividade de teste é essencial dentro do contexto da Engenharia de Software, podendo assegurar a qualidade do produto de software a partir da identificação de defeitos no software~\cite{myers2004art}. Nas últimas décadas, têm sido estabelecidas técnicas, critérios, métodos
e ferramentas para suportar o processo de desenvolvimento de software, a fim de acompanhar a crescente utilização da tecnologia por parte da população como um todo~\cite{maldonado2004introduccao}.

Existe uma larga escala de ferramentas que auxiliam na examinação do software e, a partir da examinação diretamente do software em execução, o teste fornece um \textit{feedback} de comportamentos reais, que o torna o complemento inevitável de outras técnicas de análise e garantia de qualidade na indústria. Entretanto, esta atividade ainda pode exigir muito trabalho humano, apesar da comunidade científica considerar mais a utilização de testes automatizados~\cite{bertolino2007software}.

Nos últimos 50 anos, a Engenharia de Software tem se preocupado com a influência da personalidade humana em tarefas individuais de trabalho, como aponta a revisão sistemática de literatura feita por~\cite{cruz2011personality}. Esta preocupação dá origem a pensamentos sobre como traçar estratégias para que se possa usar a análise da personalidade para a prática da Engenharia de Software. 

Os trabalhos publicados e citados sobre testes exploratórios nos leva a concluir que a personalidade humana pode ter influência sobre este método de teste~\cite{bach2003exploratory, whittaker2009exploratory, itkonen2015test, itkonen2012role, shoaib2009empirical}. As ações de testadores durante a aplicação de testes exploratórios podem variar significativamente de uma pessoa para outra, isto é, a metodologia adotada por testadores tem relação direta com os traços da personalidade de cada testador~\cite{shoaib2009empirical}. 

O trabalho de \cite{shoaib2009empirical} realizou um experimento projetado para identificar os testadores que podem alcançar o melhor resultado durante a aplicação de testes exploratórios. O trabalho se baseou em pré e pós-testes, sendo que, para testar a hipótese da pesquisa, um \textit{Teste de Aptidão de Teste Exploratório} (ETAT) foi realizado em um conjunto de testadores (amostra) para provar a influência da personalidade na aplicação dos testes. Os resultados do experimento mostram existência de uma relação positiva entre o teste exploratório e traços da personalidade humana. Em particular, os testadores que apresentaram uma personalidade extrovertida foram os mais propensos a serem bons testadores exploratórios.

As organizações estão adotando várias metodologias e forças de trabalho alternativas para realizar as entregas de software de maneira eficiente~\cite{dubey2017personas}. Traçar estratégias para que se estruture a equipe adequada para realização da atividade de teste é uma tarefa que pode reforçar estas forças de trabalho e garantir uma entrega mais ágil sem que se comprometa a qualidade do produto de software.

Embora a atividade de testes seja a solução para assegurar um produto ou serviço de software confiável, a execução dos casos de teste pode configurar um trabalho desinteressante se comparado ao projeto ou à codificação. Principalmente quando nos referimos a testes manuais exploratórios. E, sendo uma atividade baseada em humanos, os resultados de um produto de software são dependentes de fatores humanos e carregam desafios para as equipes de desenvolvimento de software, um exemplo de desafio é a busca por uma forma mais eficaz de aumentar a motivação e a satisfação dos testadores~\cite{deak2016challenges}.

Por outro lado, \cite{berner2005observations} afirma em seu trabalho que os testes automatizados nunca podem substituir completamente os testes manuais. \cite{martin2007good} também apresentou relatórios, em seu trabalho, que afirmam que os problemas relacionados ao teste de software na indústria estão relacionados com o ambiente sociotécnico e estrutura organizacional da empresa. Ou seja, os problemas enfrentados na atividade de teste não estão, necessariamente, associados apenas à causas técnicas.

A relação entre teste de software e aspecto humano foi estudada por~\cite{shah2010studying} no contexto de uma empresa baseada em serviços, onde foi observada a dimensão humana em aspectos como atitude e motivação e aspectos sociais no teste de software. Neste estudo, foram coletadas informações sobre as atitudes dos participantes em relação ao teste de software. Estas foram separadas em categorias baseadas no nível de expertise do indivíduo utilizando os níveis \textit{sênior} e \textit{júnior}. 
 
Dentro dos dois grandes grupos separados por \cite{shah2010studying}, para os profissionais \textit{sêniors}, a atividade de teste é considerada importante, porém, enfadonha, o que os fazem compartilhar com profissionais imaturos a necessidade e os benefícios de se testar. Além disso, alguns dos participantes do estudo realizado relataram a correlação entre a prática do teste de software como um aprendizado para se aprimorarem em suas atividades de desenvolvimento.

Já os profissionais \textit{juniores}, que possuíam dois anos ou menos de experiência, relataram terem adquirido um vasto aprendizado sobre o sistema no qual estavam trabalhando, visto que, após trabalharem com a atividade de testes, perceberam a importância da coerência do código e o impacto positivo que os testes podem ter na garantia da qualidade. Além disso, os resultados sobre a pesquisa mostrou que as atitudes de profissionais mais antigos podem influenciar significativamente atitudes de pessoas que não possuem uma vasta experiência. 

\cite{ekwoge2017tester} apresentou em seu trabalho vários tipos de utilizações para o termo ``experiência'' no contexto dos papéis da Engenharia de Software, incluindo: 

\begin{itemize}
    \item  \textit{Experiência do Usuário}, que está relacionada ao ponto de vista de uma pessoa diante do uso antecipado de um produto, sistema ou serviço~\cite{hassenzahl2008user};
    \item  \textit{Experiência do Desenvolvedor}, que consiste nas experiências do desenvolvedor sobre todos os tipos de artefatos e atividades relacionadas a ele~\cite{fagerholm2012developer};
    \item  \textit{Experiência do Cliente}, que baseia-se na interação do cliente com o fornecedor do produto ou serviço~\cite{palmer2010customer};
    \item \textit{Experiência de Marca}, que se refere à respostas do cliente e respostas comportamentais evocadas por estímulos relacionados que fazem parte do design e da identidade da marca, embalagem, comunicações e ambiente~\cite{brakus2009brand}.
\end{itemize}

A experiência do testador se relaciona com os artefatos que se busca testar e com às atividades de teste relacionadas a ele. Como os artefatos que serão testados são resultado de um processo de implementação realizado pela equipe de desenvolvimento, pode-se concluir que a atividade de teste é impactada, também pela experiência dos desenvolvedores que geraram estes artefatos.

O estudo de caso realizado por \cite{itkonen2015test} evidencia a experiência do testador com os TEs como uma maneira natural de se fazer testes, justificando a atividade de explorar como uma ação que enfatiza a utilização da experiência e criatividade dos testadores para encontrar defeitos durante a execução dos testes. 

Neste sentido, como já abordado por~\cite{cruz2011personality}, o desempenho da equipe pode variar a depender de seus membros e de suas respectivas características de personalidade e experiência. Esta narrativa contribui fortemente para o contexto do presente trabalho, tendo em vista que o laboratório da universidade (ITRAC) é um ambiente composto por equipes volúveis.


