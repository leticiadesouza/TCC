\begin{resumo}

A Universidade de Brasília (UnB), por meio do \textit{Information Tech-nology Research and Application Center} (ITRAC), em parceria com o Ministério da Economia (ME), tem contribuído com o Governo Federal Brasileiro para a transformação de serviços públicos em serviços digitais. O ITRAC definiu um processo de validação sistematizado com o intuito de garantir a qualidade destes serviços, a partir da utilização dos Testes Exploratórios, com a Metáfora do Turista, que dependem da personalidade e conhecimento dos envolvidos na atividade de teste. Dessa forma, supõe-se que o perfil do testador tenha algum impacto na aplicação das \textit{tours} utilizadas na metáfora do turista. Neste trabalho propomos a obtenção de perfis de testadores, utilizando características pessoais e experiência do domínio para alocação automática de atividades de teste, utilizando a equipe de testadores do ITRAC. O principal procedimento técnico escolhido para esta avaliação foi a Pesquisa-Ação, comumente utilizada na resolução de problemas coletivos, envolvendo pesquisadores e participantes de modo cooperativo. A primeira etapa desta metodologia, o Diagnóstico, foi realizado para compreensão do objeto de estudo, enquanto que as três últimas etapas, o Planejamento, a Ação e a Avaliação serão levadas em consideração para a construção de instrumentos em ciclos de interação com a equipe, afim de facilitar o entendimento sobre o funcionamento do módulo da ferramenta proposta.

 \vspace{\onelineskip}
    
 \noindent
 \textbf{Palavras-chave}: latex. abntex. editoração de texto.
\end{resumo}
    